%!TEX root = ../aluno.tex

\thispagestyle{empty}
{\small


\begin{center}
Projeto: LIVRO ABERTO DE MATEMÁTICA

\begin{tabular}{m{.25\linewidth}m{.25\linewidth}m{.25\linewidth}}
\includegraphics[scale=.15]{impa} & \includegraphics[width=3cm]{logo.jpg} & \includegraphics[scale=.24]{obmep}
\end{tabular}

\url{umlivroaberto.org}

\vfill
\renewcommand{\arraystretch}{1.325}

\begin{tabular}{p{.175\textwidth}p{.75\textwidth}}
Título & Frações no Ensino Fundamental - Volume 1\\
Ano/ Versão & 2021 / versão 3.0 de Fevereiro de 2021 \\

Editora & Instituto Nacional de Matem\'atica Pura e Aplicada (IMPA-OS)\\
Realização& Olimp\'iada Brasileira de Matem\'atica das Escolas P\'ublicas (OBMEP)\\
Produção& Livro Aberto\\

Coordenação & Fabio Simas e Augusto Teixeira\\

Autores & Cydara Cavedon Ripoll, Fabio Luiz Borges Simas, Humberto José Bortolossi, Letícia Guimarães Rangel, Victor Augusto Giraldo, Wanderley Moura Rezende, Wellerson  da Silva Quintaneiro\\

Colaboradores & Ana Paula Pereira (CAp UFF), Andreza Gonçalves (estudante da UFF), Bruna Luiza Oliveira (estudante da UFF), Francisco Mattos (Colégio Pedro II), Helano Andrade (estudante da UNIRIO), João Carlos Cataldo (CAp UERJ e Colégio Santo Ignácio), Luiz Felipe Lins (Secretaria de Educação da Cidade do Rio de Janeiro), Michel Cambrainha (UNIRIO), Rodrigo Ferreira (estudante da UNIRIO), Tahyz Pinto (estudante da UFF) \\

Coordenação \newline de arte  & José Ezequiel Soto Sánchez \\

Projeto gráfico & Enzo Esberard \\

Diagramação & Tarso Boudet Caldas \\

Ilustração \newline Artística: & Agnes Antonello e Aline Santiago \\

Ilustração \newline Técnica & 
Briza Aiki Matsumura,
Caio Felipe da Silva Evangelista,
Eduardo Filipe de Miranda Souto,
Gisela Alves de Souza,
Israel Fialho Magalhães, 
Kayky Zigart Carlos,
Livia Machado da Silveira Verly,
Luiz Fernando Alves Macedo,
Lucas Hideo Maekawa,
Lucas Oliveira Machado de Sousa,
Mauricio de Azevedo Neto,
Maurício Menegatti Andrade,
Tarso Boudet Caldas,
Vitoria da Mota Souza,
Vinícius Marcondes de Paula Silva e 
Wanessa Souza de Oliveira\\

Capa & Fabio Simas
\end{tabular}
\vspace{.2cm}

\includegraphics[height=30pt]{cc}
%\vspace{.3cm}

Após o dia $1^{\textrm{\underline{o}}}$ de setembro de 2026 esta obra passa a estar licenciada por CC-by-sa.

%Algumas figuras podem possuir licença com mais direiatos do que a vigente para todo o material.
\end{center}
}
\setcounter{page}{1}
% \thispagestyle{empty}
\cleardoublepage

